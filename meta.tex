% -----------------------------------------------------------------------------
%                                      PACOTES PRINCIPAIS
% -----------------------------------------------------------------------------

%% Codificação e formatação básica do LaTeX
% Dimensoes da pagina
\usepackage[paperheight=297mm,paperwidth=210mm,top=3cm,left=3cm,right=2cm,bottom=2cm]{geometry}    % Dimensões da página (A4)
\usepackage[portuges, brazilian]{babel} % Hiphenação em portugues
\usepackage[utf8]{inputenc}             % Codificação do arquivo
\usepackage[T1]{fontenc}
\usepackage{uarial}                      % Pacote de fonte Times
\usepackage{lhelp}                      % Denine macros úteis
\usepackage{ccaption}                   % Opções para capítulos
\usepackage{indentfirst}                % Identação do primeiro parágrafo
\usepackage{fancyhdr}                   % Controlar os cabeçalhos e rodapés
\usepackage{enumerate}                  % Para usar enumerações
\usepackage{cmap}                       % Mapear caracteres especiais no PDF

\usepackage{floatflt}                   % Quebra de texto em figuras e tabelas
\usepackage{colortbl}                   % Adicionar cor às tabelas
\usepackage{fancyvrb}                   % Texto verbatim sofisticado
\usepackage{wrapfig}                    % Produz figuras que o texto pode ficar ao redor
\usepackage{url}                        % Adiciona URLs
\usepackage{multirow}                   % Permite construir células multi-linhas

% Essencial para colocar funções e outros símbolos matemáticos
\usepackage{amsmath, amssymb, amsthm}
\usepackage{amscd, amsfonts, textcomp}
\usepackage{psfrag,layout}

% Links dinâmicos
% Suporte para hipertexto, links para referências e figuras

% To Dvi.
%\usepackage[dvipdfmx,plainpages=false,pdfpagelabels,colorlinks=true,
%citecolor=black,linkcolor=black,urlcolor=black,filecolor=black,
%bookmarksopen=true]{hyperref}

% To pdflatex
\usepackage[pdftex, plainpages=false, pdfpagelabels, colorlinks=true,
citecolor=black,linkcolor=black, urlcolor=black, filecolor=black,
bookmarksopen=true]{hyperref}

\usepackage[dvipsnames,svgnames,table]{xcolor}
\usepackage{pdfpages}			% Inclui paginas em pdf no arquivo
\usepackage{epstopdf}			% Converte figuras em formato eps para pdf
\usepackage{color}				% Suporte para cores
\usepackage{transparent}		% Suporte para vários conjuntos de cores
\usepackage{siunitx}			% Sistema internacional de unidades
\usepackage{lipsum}

\usepackage[all]{hypcap}                % Soluciona o problema com o hyperref e capítulos
\usepackage{wasysym}                    % Símbolos especiais com \female e \male etc.
\usepackage{bigstrut}                   % Espaçamento na tabela
\usepackage{comment}                    % Comentar várias linhas simultâneas
\usepackage{epigraph}                   % Epígrafe
\usepackage[titles,subfigure]{tocloft}  % Permite formatar o sumário
\usepackage{makeidx}
\usepackage{emptypage}					 % Retira o número de páginas em branco
\usepackage{afterpage}
\usepackage{kantlipsum}					 % mock text
\usepackage{setspace}                   % Para definir espaçamento entre as linhas
\usepackage{listings}                   % Para Inserir código fonte
\usepackage{mathtools, nccmath}         % Para alinhamento e redução na fonte de equações

% -----------------------------------------------------------------------------
%                                       ELEMENTOS GRÁFICOS
% -----------------------------------------------------------------------------
\usepackage{graphics}                   % Suporte padrão para gráficos
\usepackage{graphicx}                   % Suporte avançado para gráficos
\usepackage[]{graphicx}                 % Para incluir figuras (pacote extendido)
\usepackage[normalsize]{subfigure}      % Possibilita a utilização de subfiguras
% \usepackage{subfig}                     % Criar figura dividida em subfiguras
\usepackage{epsfig}                     % Uso de figuras eps.
\usepackage{epsf}

\usepackage{caption}                    % Customizar as legendas de figuras e tabelas
\usepackage{multicol}                   % Criar ambientes com 2 ou mais colunas
% -----------------------------------------------------------------------------
% Pacotes para Tabelas
\usepackage{array}                      % Elementos extras para formatação de tabelas
\usepackage{booktabs}                   % Tabelas com qualidade de publicação
\usepackage{longtable}                  % Para criar tabelas maiores que uma página
\usepackage{lscape}                     % adicionar tabelas e figuras como landscape

%% Lista de Abreviações
% \usepackage[notintoc,portuguese]{nomencl}   % Cria lista de abreviações
\usepackage{nomencl}
\makenomenclature

% %  Lista de Tabela de siglas e simbolos.
%\usepackage{tabela-simbolos} 
% %  Letras maiúsculas são colocados antes dos símbolos de minúsculas.
%\usepackage[caixa=Mm]{tabela-simbolos}
% %  Define a ordem em que vai aparecer os simbolos, as letras gregas e romanas.
%\usepackage[romanos=2,gregos=3,simbolos=1]{tabela-simbolos}

%% Notas de rodapé
\usepackage{footnote}                   % Lidar com notas de rodapé em diversas situações
\makesavenoteenv{tabular}               % Notas criadas nas tabelas ficam no fim das tabelas
\usepackage{pifont}                     % Acesso a PostScript Symbol padrão e fontes Dingbats
\usepackage{lastpage}                   % Conta o número de páginas
\usepackage{ifoddpage}				 % Inclui n de páginas impares.
\newcommand\alwaysodd[1]{%
  \checkoddpage
  \ifoddpage
    #1%
  \else
    \mbox{}\clearpage#1%
  \fi
}

%% Referências bibliográficas e afins
%\usepackage[round]{natbib}              % Citações tipo (nome-ano)
%\usepackage{natbib}                    % Citações tipo [nome-ano]

% Adicionar bibliografia, índice e conteúdo na Tabela de conteúdo
% Não inclui lista de tabelas e figuras no índice
\usepackage[nottoc,notlof,notlot]{tocbibind}

%% Pontuação e unidades
\usepackage{icomma}		% Posicionar inteligentemente a vírgula como separador decimal
\usepackage[tight]{units}	% Formatar as unidades com as distâncias corretas
% -----------------------------------------------------------------------------
%                                   FORMATAÇÕES DE TEXTO
% -----------------------------------------------------------------------------
\usepackage{parskip}					% Espaçamento entre paragrafos
%\special{papersize=210mm,297mm}    		% Tamanho do papel (neste caso papel A4)
\pagenumbering{roman}                   % Numeração das paginas iniciais em romano
\setlength{\parindent}{1cm}             % Parágrafo de 1cm
% \setlength{\mathindent}{1cm}          % Indentação das fórmulas matemáticas
\setlength{\fboxsep}{1mm}               %   Espaçamento do texto para o frame, também define
                                        % espaçamento entre colunas da tabela
                                        
\frenchspacing                          % Não põe um espaço adicional após ponto final
\sloppy                                 % Força que todas as linhas fiquem dentro das margens
\raggedbottom                           % Para não permitir espaços extra no texto
\def\figurename{Figura.}                % Define como a palavra figura será escrito
\def\tablename{Tabela.}                 % Define como a palavra tabela será escrito

\newcommand{\eq}[1]{Equação~(\ref{#1})}	% Facilita a citação de Equações
\newcommand{\figu}[1]{Figura~\ref{#1}}		% Facilita a citação de Figuras
\newcommand{\tab}[1]{Tabela~\ref{#1}}		% Facilita a citação de Tabelas

% Citação tipo (AUTOR, ano), conforme ABNT
%\renewcommand{\cite}[1]{{\citeauthor{#1},~\citeyear{#1}}}
%\renewcommand{\citep}[1]{(\citeauthor{#1},~\citeyear{#1})}
%\renewcommand{\citet}[1]{\citeauthor{#1}}

%
\theoremstyle{definition}
\newtheorem{defi}{Definição}[chapter]		% Para introduzir uma definição
\newtheorem{teo}{Teorema}[chapter]		% Para introduzir um teorema
\newcommand{\nomunit}[1]{%
\renewcommand{\nomentryend}{\dotfill[#1]}}	% Colocar unidades na lista de Abreviaturas e siglas

%% Estes comandos criam os subgrupos na Lista de Abreviaturas e Siglas
\renewcommand{\nomgroup}[1]{%
  \ifthenelse{\equal{#1}{A}}{\item[\textbf{Letras Latinas}]}{\hrulefill%			a
    \ifthenelse{\equal{#1}{C}}{\item[\textbf{Letras Gregas}]}}}}				c
        \ifthenelse{\equal{#1}{G}}{\item[\textbf{Subscritos}]}}}}}}			e
           \ifthenelse{\equal{#1}{K}}{\item[\textbf{Siglas}]}{}}{}}{}}{}}{}}{}}%		f
%
\setcounter{secnumdepth}{5}	% Serve pra introduzir um quinto nível de seções (subsubsubsection)
\setcounter{tocdepth}{5}	% Serve pra introduzir um quinto nível de seções (subsubsubsection)
%
\makeatletter			% Changes the category code of '@' character to 11
% -----------------------------------------------------------------------------
%                                          OPERADORES
% -----------------------------------------------------------------------------
% Conforme a necessidade de cada trabalho
\newcommand{\vectornorm}[1]{\left|\left|#1\right|\right|}   % Facilita a introdução da norma ||w||
\newcommand{\argmax}{\operatornamewithlimits{argmax}}       % Facilita a escrita de argmax
\newcommand{\latex}{\LaTeX~}                                % Facilita a escrita de LaTeX
\newlength\figureheight
\newlength\figurewidth
% -----------------------------------------------------------------------------
%                             FORMATANDO SEÇÕES, TABELAS E EQUAÇÕES
% -----------------------------------------------------------------------------
\@addtoreset{figure}{chapter}
\renewcommand{\thefigure}{\thechapter.\@arabic\c@figure}

\@addtoreset{equation}{chapter}
\renewcommand{\theequation}{\thechapter.\@arabic\c@equation}

\@addtoreset{table}{chapter}
\renewcommand{\thetable}{\thechapter.\@arabic\c@table}

\def\chapter{\@startsection{chapter}{1}
          {\z@}{26pt minus 0pt}{12pt minus 0pt}{\renewcommand{\rmdefault}{phv}\large\bf}}
\def\appendix{\@startsection{chapter}{1}
          {\z@}{26pt minus 0pt}{12pt minus 0pt}
          {\large{\textsf{\textbf{APÊNDICE}}}~~\renewcommand{\rmdefault}{phv}\large\bf}}
\def\section{\@startsection{section}{2}
          {\z@}{26pt minus 0pt}{12pt minus 0pt}{\renewcommand{\rmdefault}{phv}\bf}}
\def\subsection{\@startsection{subsection}{3}
          {\z@}{26pt minus 0pt}{12pt minus 0pt}{\renewcommand{\rmdefault}{phv}\bf}}
\def\subsubsection{\@startsection{subsubsection}{4}
          {\z@}{26pt minus 0pt}{12pt minus 0pt}{\renewcommand{\rmdefault}{phv}\bf}}
\def\subsubsubsection{\@startsection{subsubsubsection}{5}
          {\z@}{26pt minus 0pt}{12pt minus 0pt}{\renewcommand{\rmdefault}{phv}\bf}}
         
\newcounter {anexo}[chapter]
\def\anexo{\@startsection{chapter}{1}
          {\z@}{26pt minus 0pt}{12pt minus 0pt}
	  {\large{\textsf{\textbf{ANEXO}}}~~\renewcommand{\rmdefault}{phv}\large\bf}}

% -----------------------------------------------------------------------------
%                                         SUB-SUB-SUB-SECTION
% -----------------------------------------------------------------------------
\newcounter {subsubsubsection}[subsubsection]
\renewcommand\thesubsubsubsection{}
% \newcommand\subsubsubsection{\@startsection{subsubsubsection}{4}{\z@}%
%                             {12pt minus 0pt}{12pt minus 0pt}{\normalfont\normalsize\it}}
% Espaçamento do numero da subsubsubseção em relação à marge no sumário.
\newcommand*\l@subsubsubsection{\@dottedtocline{3}{8em}{2.2em}}
\newcommand*{\subsubsubsectionmark}[1]{}

% -----------------------------------------------------------------------------

\def\thechapter{\arabic{chapter}}
\def\thesection{\thechapter.\arabic{section}}	
\def\thesubsection{\thesection.\arabic{subsection}}
% \def\thesubsubsection{\Alph{subsubsection}}
\def\thesubsubsection{}
\makeatother				% Reverts (\makealetter) this to its original catcode of 12.

%\renewcommand\cftsecfont{\bf}                % Títulos das seções em negrito no sumário
%\renewcommand{\cftsecleader}{\cftdotfill{\cftnodots}}  % Elimina os pontos na frente das seções no sumário
%\setlength\cftbeforesecskip{12pt}            % Espaçamento antes da seção no sumário
%\renewcommand{\cftsecindent}{0 em}           % Espaçamento do numero da seção em relação à marge no sumário
%\renewcommand{\cftsecnumwidth}{1.9 em}       % Espaçamento do título da seção em relação ao número no sumário
%\renewcommand{\cftsubsecindent}{1.9 em}      % Espaçamento do numero da subseção em relação à marge no sumário
%\renewcommand{\cftsubsecnumwidth}{2.8 em}    % Espaçamento do título da subseção em relação ao número no sumário
%\renewcommand{\cftsubsubsecindent}{4.7 em}   % Espaçamento do numero da subsubseção em relação à marge no sumário
\renewcommand{\cftsubsubsecnumwidth}{1.9 em}  %     Espaçamento do título da subsubseção em relação
                                              % ao número no sumário



% -----------------------------------------------------------------------------
%                                     ESTILOS DO PACOTE FANCYHDR
% -----------------------------------------------------------------------------
% Usar os estilos do pacote fancyhdr
\pagestyle{fancy}
\fancypagestyle{plain}{\fancyhf{}}
\fancyhead{}                                % Limpar os campos do cabeçalho atual
% Número da página do lado esquerdo [L] nas páginas ímpares [O] e do lado direito [R] nas páginas pares [E]
%\fancyhead[LO,RE]{\thepage}
%\fancyhead[RO]{\nouppercase{\rightmark}}   % Nome da seção do lado direito em páginas ímpares
%\fancyhead[LE]{\nouppercase{\leftmark}}    % Nome do capítulo do lado esquerdo em páginas pares
\fancyfoot{}                                % Limpar os campos do rodapé
\fancyfoot[CO, CE]{\thepage}
\renewcommand{\headrulewidth}{0pt}          % Omitir linha de separação entre cabeçalho e conteúdo
\renewcommand{\footrulewidth}{0pt}          % Omitir linha de separação entre rodapé e conteúdo
\headheight 14.5pt                          % Altura do cabeçalho



% -----------------------------------------------------------------------------
%                                        COMANDOS CUSTOMIZADOS
% -----------------------------------------------------------------------------
%  Ambiente para criar exemplos dentro de um capítulo.
\usepackage{verbatim}
\usepackage{float}
\floatstyle{boxed}
\newfloat{program}{!th}{lop}
\floatname{program}{Exemplo}

\newbox\examplebox
\newenvironment{example}[2]{%
    \program
    \vskip-.7\baselineskip
    \caption{#2}%
    \label{#1}%
    \verbatim
}{%
    \endverbatim
    \vskip-.7\baselineskip
    \endprogram
}
\numberwithin{program}{chapter}

%% Redefinição de marcadores.
% Types of itemize
% \circ — An open circle
% \cdot — A centered dot
% \star — A five-pointed star
% \ast — A centered asterisk
% \rightarrow — A short right-pointing arrow
% \diamondsuit — An open diamond
\renewcommand{\labelitemi}{$\circ$}
\renewcommand{\labelitemii}{$\cdot$}
\renewcommand{\labelitemiii}{$\diamond$}
\renewcommand{\labelitemiv}{$\ast$}


