% -----------------------------------------------------------------------------
%                                CAPA DO TRABALHO
% -----------------------------------------------------------------------------

%Quando for imprimir, mude os comandos de \clearpage para \cleardoublepage, pois assim quando mandar pra impressora não terá problemas ao imprimir frente e verso (basta apagar \clearpage e habilitar o \cleardoublepage). 

\thispagestyle{fancy}
\fancyhf{}
\lhead{
  \begin{picture}(225,150)%(horizontal,vertical)
    \put(0,0){\includegraphics[scale=0.13]{pre/logo-unicamp.jpg}}
  \end{picture}
}
\cfoot{} 

\begin{center}
\vspace*{5cm}
{\fontsize{14}{14} \textbf{UNIVERSIDADE ESTADUAL DE CAMPINAS}}	\\ 
\vspace{0.6ex}
{\fontsize{14}{14} \textbf{Faculdade de Engenharia Mecânica}}	\\ 
\vspace{3cm}
%{\fontsize{12}{12} \textbf{Relatório Parcial}}	\\ 
{\fontsize{12}{12} \textbf{Trabalho de Conclusão de Curso I}}	\\
\vspace{3cm}
{\fontsize{23}{23} \textbf{Simulação da Mão Humana Utilizando o Modelo de Hill Simplificado}}\\
\vspace{1.6ex}
{\fontsize{23}{23} \textbf{}}\\ 
\end{center}
\vspace{3cm}
\begin{flushright}
{\fontsize{12}{12} \textbf{Autor: Giovani Valencise}}\\ 
{\fontsize{12}{12} \textbf{Orientador: Prof. Dr. Eric Fujiwara}}\\
\end{flushright}
\vspace{2cm}
\begin{center}
CAMPINAS\\ 
2019
\end{center}


% -----------------------------------------------------------------------------
%                              ASSINATURA DO ORIENTADOR
% -----------------------------------------------------------------------------
%\cleardoublepage - habilite para imprimir
\clearpage
\thispagestyle{plain}



\begin{center}
\vspace*{0.8cm}

{\fontsize{14}{14} \textbf{UNIVERSIDADE ESTADUAL DE CAMPINAS}}	\\
{\fontsize{14}{14} \textbf{FACULDADE DE ENGENHARIA MECÂNICA}}	\\

\vspace{3cm}
%{\fontsize{12}{12} \textbf{Relatório Parcial}}	\\ 
{\fontsize{12}{12} \textbf{Trabalho de Conclusão de Curso I}}	\\
\vspace{3cm}
{\fontsize{23}{23} \textbf{Simulação da Mão Humana Utilizando o Modelo de Hill Simplificado}}\\
\end{center}

\vspace*{2cm}
\begin{flushleft}
Trabalho de Conclusão de Graduação apresentada à Faculdade de Engenharia Mecânica da Universidade Estadual de Campinas como parte dos requisitos exigidos para obtenção do título de Engenheiro no curso de Engenharia de Controle e Automação.
\end{flushleft}

\vspace*{0.5cm}



\vspace{0.5cm}
\noindent
Autor: Giovani Valencise\\
Orientador: Prof. Dr. Eric Fujiwara	\\


\vspace{0.6cm}

\vspace{4.0cm}
\begin{center}
\textbf{CAMPINAS}\\ 
\textbf{2018}
\end{center}


% % -----------------------------------------------------------------------------
% %                                     FICHA CATALOGRÁFICA
% % -----------------------------------------------------------------------------
% \newpage %essa parte deve estar atrás da contra capa, logo não precisa colocar o comando \cleardoublepage

% \begin{center}
% \begin{tabular}{c}
%   FICHA CATALOGRÁFICA ELABORADA PELA \\
%   BIBLIOTECA DA ÁREA DE ENGENHARIA E ARQUITETURA - BAE - UNICAMP \\
% \end{tabular}
% \end{center}

% \vspace{0.8cm}
% \begin{tabular}{|cl|} 
% 	\hline
% 	&Agência(s):\\
% 	&Nº do Proc.:\\
% 	\hline	
% \end{tabular}
%-----------------------------------------------------------------------------
% Essa parte é feita pela BAE, você terá que incluir a página feita por eles com o comando "input" pdf. A tabela seguinte é um exemplo. Veja a observação 2 antes de retirar o comentario. Note que não foi centralizada, para fazer isso e so deslocar o comando \end{center} acima. 

%\begin{tabular}{|cl|} \hline
%  \hspace{2cm} & \\
%  & Penteado, Marcos Roberto Mendes, 1985-				\\  
%  P387t & \hspace{0.15in} Transporte de grãos por leito móvel em um escoamento turbulento :\\
%  & deslocamento de grãos individuais / Marcos Roberto Mendes Penteado. –\\
%  & Campinas, SP : [s.n.], 2015.\\
%  &  \\  
%  & \hspace{0.15in} Orientador: Erick de Moraes Franklin.\\
%  & \hspace{0.15in} Dissertação (mestrado) – Universidade Estadual de Campinas, Faculdade de\\
%  & Engenharia Mecânica. \\
%  & \\
%  & \hspace{0.15in} 1. Transporte de sedimentos. 2. Escoamento turbulento. 3. Tratamento de\\
%  & imagens. I. Franklin, Erick de Moraes,1974-. II. Universidade Estadual de\\
%  & Campinas. Faculdade de Engenharia Mecânica. III. Título.\\
%  & \\ \hline
%\end{tabular}
%-----------------------------------------------------------------------------

% \vspace{1.8cm}
% \textcolor{Blue}{Obs. 1. Quando se tratar de Teses e Dissertações financiadas por agências de fomento, os beneficiados deverão fazer referência ao apoio recebido e inserir esta informação na Ficha Catalográfica, além do nome da agência, o número do processo pelo qual recebeu o Auxílio.}\\

% \vspace{0.8cm}
% \textcolor{Blue}{Obs 2. Alunos bolsistas favor solicitarem esse número na CPG/FEM.}\\

% \vspace{0.8cm}
% \textcolor{Blue}{Obs. 3. Caso a tese de doutorado seja feita em Cotutela, será necessário informar na ficha catalográfica o fato, a Universidade convenente, o País e o nome do Orientador/Coorientador.}\\

\fancyhead{}	% Limpa os campos do cabeçalho atual

% % -----------------------------------------------------------------------------
% %                                     FOLHA DE APROVAÇÃO
% % -----------------------------------------------------------------------------
% %\cleardoublepage
% \clearpage
% \begin{center}


% {\large\textbf{UNIVERSIDADE ESTADUAL DE CAMPINAS\\\vspace{1.2ex}
% FACULDADE DE ENGENHARIA MECÂNICA\\\vspace{1.2ex}
% COMISSÃO DE GRADUAÇÃO EM ENGENHARIA DE CONTROLE E AUTOMAÇÃO\\\vspace{1.2ex}
% }}

% \vspace{0.7cm}
% \textbf{TESE DE CONCLUSÃO DE GRADUAÇÃO}

% \vspace{1.2cm}
% {\fontsize{23}{23} \textbf{Aplicação de redes neurais e aprendizado de máquina para agrupamento de bases de dados de forma inteligente}}\\
% \vspace{1.6ex}
% {\fontsize{23}{23} \textbf{}}\\ 

% \begin{flushleft}
% \vspace{0.4cm}
% Autor: Bruno de Souza Ferreira\\
% \vspace{0.2cm}
% Orientador: Prof. Dr. Eric Fujiwara\\


% \vspace{0.3cm}
% A Banca Examinadora composta pelos membros abaixo aprovou esta Tese:  \\

% \vspace{0.5cm}
% \noindent\rule{10cm}{0.4pt}\\
% \textbf{Prof. Dr.\\
% Instituição \\}

% \vspace{0.4cm}
% \noindent\rule{10cm}{0.4pt}\\
% \textbf{Prof. Dr.\\
% Instituição \\}

% \vspace{0.4cm}
% \noindent\rule{10cm}{0.4pt}\\
% \textbf{Prof. Dr.\\
% Instituição \\}

% \vspace{0.4cm}
% \noindent\rule{10cm}{0.4pt}\\
% \textbf{Prof. Dr.\\
% Instituição \\}

% \vspace{0.4cm}
% \noindent\rule{10cm}{0.4pt}\\
% \textbf{Prof. Dr.\\
% Instituição \\}
% \end{flushleft}

% \begin{flushright}
% \vspace{1.2cm}
% Campinas, XX de XXXXXXXXX de 20XX.
% \end{flushright}

% \end{center}

% -----------------------------------------------------------------------------
%                                         DEDICATÓRIA
% -----------------------------------------------------------------------------
% %\cleardoublepage
% \clearpage
% \begin{center}
% \chapter*{Dedicatória}
% \end{center}
% \vspace*{1cm}

% Dedico este trabalho de graduação ...


% -----------------------------------------------------------------------------
%                                       AGRADECIMENTOS
% -----------------------------------------------------------------------------
%\cleardoublepage
% \clearpage
% \begin{center}
% \chapter*{Agradecimentos}
% \end{center}

% \vspace*{1cm}
% \begin{trivlist}  \itemsep 2ex  \normalsize

% \item Ao meu orientador, Prof. Dr. Eric Fujiwara,



% \end{trivlist}


% % -----------------------------------------------------------------------------
% %                                         EPÍGRAFE
% % -----------------------------------------------------------------------------
% %\cleardoublepage
% \clearpage
% \vspace*{8in}
% \epigraph{\emph{Algo só é impossível até que alguém duvide e resolva provar ao contrário}}{Albert Einstein}


% -----------------------------------------------------------------------------
%                                          RESUMO
% -----------------------------------------------------------------------------
%\cleardoublepage
% \clearpage

% \begin{center}
% \chapter*{Resumo}
% \end{center}
% \vspace{24pt}
% \onehalfspacing
% \noindent
% \textcolor{Blue}{Escrever o resumo com no máximo 500 palavras.}\\

% \vspace{1cm}
% \noindent
% \emph{Palavras-chave}: Palavra 1, Palavra 2, etc.
% \\

%-----------------------------------------------------------------------------
%                                        ABSTRACT
% -----------------------------------------------------------------------------
% %\cleardoublepage
% \clearpage
% \begin{center}
% \chapter*{Abstract}
% \end{center}
% \vspace{24pt}
% \onehalfspacing
% \noindent
% \textcolor{Blue}{Write the abstract here, no more than 500 words.}\\\\

% \vspace{1cm}
% \noindent
% \emph{Keywords}: Word 1, Word 2, ... .

%-----------------------------------------------------------------------------
%                                    LISTA DE ILUSTRAÇÕES
% -----------------------------------------------------------------------------
% \cleardoublepage
\clearpage
\renewcommand*\listfigurename{Lista de Ilustrações}

\begin{center}
\listoffigures
\end{center}
%-----------------------------------------------------------------------------
%                                    LISTA DE TABELAS
% -----------------------------------------------------------------------------
%\cleardoublepage
% \clearpage
% \begin{center}
% \listoftables
% \end{center}
%-----------------------------------------------------------------------------
%                             LISTA DE ABREVIATURAS E SIGLAS
% -----------------------------------------------------------------------------
%\cleardoublepage
% \clearpage
% \clearpage
\vspace*{1cm}
\begin{center}
	\chapter*{Lista de Abreviaturas e Siglas}
\end{center}

\vspace{1cm}
\noindent
\textbf{\emph{Letras Latinas}}\\\\
\noindent
\begin{tabular}{l c p{.6\linewidth} c }

	$A$ & - & área da seção transversal & [$m^2$]\\
	$A_{field}$ & - & área da seção filmada & [$m^2$]\\
	$b$ ou $L$  & - & largura do canal & [$m$]\\
	$C_D$ & - & fator de arrasto & [-] \\
	$d$ & - & diâmetro do grão & [$\mu m$]\\
	$d_{50}$ & - & diâmetro médio das partículas & [$\mu m$]\\
	$D_h$ & - & diâmetro hidráulico & [$m$] \\
	$g$ & - & aceleração da gravidade local & [$m/s^2$]\\
	$h$ & - & altura do canal & [$m$]\\
	$L_e$ & - & comprimento de entrada & [$m$]\\
	$P$ & - & perímetro molhado & [$m$]\\	
	$Q$ & - & vazão de água & [$m^3/s$]\\
	$Q_B$ & - & vazão de grãos & [$m^3/s$]\\
	$r$ & - & profundidade do escoamento em relação a superfície & [$m$]\\
	$Re$ & - & número de Reynolds & [-]\\
	$Re_*$ & - & número de Reynolds na escala da partícula & [-]\\
	$t$ & - & tempo & [$s$] \\
	$u$ & - & componente de velocidade na direção x & [$m/s$]\\
	$\bar{U}$ & - & velocidade média do escoamento & [$m/s$]\\
	$\bar{U_p}$ & - & velocidade média da partícula & [$mm/s$]\\
	$u_*$ & - & velocidade de atrito & [$m/s$]\\
	$v$ & - & componente de velocidade na direção y & [$m/s$]\\
	$\bar{V}$ & - & velocidade média de escoamento & [$m/s$]\\

\end{tabular}
\clearpage
\noindent
\textbf{\emph{Letras Gregas}}\\\\
\noindent
\begin{tabular}{l c p{.6\linewidth} r}

	$\epsilon^+$ & - & Rugosidade relativa & [-] \\
	$\epsilon$ & - & Rugosidade da superfícies (leito) & [$\mu m $] \\
	$\mu$ & - & viscosidade dinâmica & [$Pa.s$]\\
	$\nu$ & - & viscosidade cinemática & [$m^2/s$]\\
	$\rho$ & - & massa específica & [$kg/m^3$]\\
	$\tau$ & - & tensão de cisalhamento & [$N/m^2$] \\
	$\theta$ & - & Número de Shields & [-] \\

\end{tabular}
% -----------------------------------------------------------------------------
\newline \newline
\textbf{\emph{Siglas}}\\\\
\noindent
\begin{tabular}{l c p{.8\linewidth} }

	CCD & - & Charge-Coupled Device\\
	CPO & - & Campo de imagem\\
	FEM & - & Faculdade de Engenharia Mecânica \\
	LED & - & Light-Emitting Diode\\
	LMF & - & Laboratório de Mecânica dos Fluidos\\
	NS & - & Navier-Stokes\\
	PIV & - & Particle Image Velocimetry\\
	TFD & - & Tamanho, forma e densidade \\
	UNICAMP & - & Universidade Estadual de Campinas\\
	
% -----------------------------------------------------------------------------
\end{tabular}
\newline \newline
\textbf{\emph{Subscritos}}\\\\
\noindent
\begin{tabular}{l c p{.6\linewidth} r}

	c ou th & - & valor crítico (threshold)& \\
	f & - & fluido & \\
	s & - & sólido & \\
	
\end{tabular}
% -----------------------------------------------------------------------------
\newline \newline
\textbf{\emph{Outras Notações}}\\\\
\noindent
\begin{tabular}{l c p{.8\linewidth} }

	1D & - & Elemento unidimensional\\
	2D & - & Elemento bidimensional\\
	3D & - & Elemento tridimensional\\
	
\end{tabular}
 %inicia o arquivo de simbolos.tex 

% -----------------------------------------------------------------------------
%                                         SUMÁRIO
% -----------------------------------------------------------------------------
%\cleardoublepage
\clearpage
\setcounter{page}{23}

\begin{center}
\tableofcontents
\end{center}
%\clearpage

%-----------------------------------------------------------------------------
%                                    DEMAIS FORMATAÇÕES
% -----------------------------------------------------------------------------
% Espaçamento de 1.5
\onehalfspacing     

% Inicialização de capítulos apenas em páginas impares
\let\originalchapter=\chapter
\def\chapter{\cleardoublepage\originalchapter}

% Mudando a numeração de páginas para o formato arábico

\pagenumbering{arabic}


