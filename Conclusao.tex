\selectlanguage{Brazilian}

\chapter{CONCLUSÃO}\label{cap5}
Este projeto visava mostrar a relação entre o modelo de Hill descrito por \cite{hill1938heat} e a cinemática direta da mão de um ser humano. O projeto baseou-se em trabalhos como os de \cite{zajac1989muscle}, \cite{lee1995model}, \cite{feng1999surface} e \cite{rosen1999performances}, para analisar e validar a integração entre as ativações musculares e a cinemática da mão.

Analisando os resultados obtidos no capítulo \ref{cap4} pode-se dizer que as integrações entre os modelos se deu de forma satisfatória, já que os músculos respondiam corretamente aos níveis de atuação e durante as simulações era possível notar o crescimento da velocidade conforme os músculos eram excitados. Todas as validações no capítulo \ref{cap4} valem para os outros dedos da mão e do dedo polegar, bastando corrigir os valores de MJG para os diferentes de dedos, e no caso do polegar, ainda, as matrizes de DH são diferentes devido aos parâmetros de DH serem diferentes. Sendo assim a vantagem do modelo obtido neste projeto reside no nível de atuação, e os parâmetros do músculo estarem normalizados, portanto as respostas se limitam as restrições físicas do dedo e parâmetros como MJG.

O objetivo futuro para este estudo consiste em fazer uma modelagem mais completa da mão levando em consideração outras restrições como descritas em \cite{lee1995model}, e como os dedos interagem com a palma da mão, onde existem novos graus de liberdade para todos os dedos. E aprimorar o modelo de Hill construído para que este fique mais robusto.

