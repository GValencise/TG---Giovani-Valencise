\selectlanguage{Brazilian}

\chapter{INTRODUÇÃO}\label{cap1}

%%ENTRA O FINAL, QUE É A PROPOSTA

Os movimentos de maior precisão realizados pelos seres humanos são executados pelas mãos, como, por exemplo, pinçamento ou pressão. É assim que o ser humano interage com o ambiente e o transforma, durante centenas de anos. O trabalho manual já é parte da rotina do humano e tem, cada vez mais, sido objeto de estudo, já que o número de combinações possíveis de movimento é expressivo. A maior parte das ferramentas desenvolvidas pelos humanos foram adaptadas para o uso manual de forma a auxiliar nas tarefas executadas e dar maior autonomia e performance para o executor \cite{oliveira2016modelagem} \cite{gustus2012human}. A compreensão dos movimentos e como eles interagem com estas ferramentas é essencial para se construir modelos e tecnologias que auxiliem em cenários como a indústria, área médica e aplicações aeroespaciais \cite{rosen1999performances}. Os dedos são parte fundamental da movimentação das mãos e na desenvoltura com que as atividades são realizadas, eles são tão vitais que um indivíduo que não possui dedos é considerado 54\% menos capaz que uma pessoa com todos os dedos intactos.\cite{engelberg1988guides} \cite{oliveira2016modelagem}.

No presente trabalho será realizado a implementação de um modelo de ativação muscular, a fim de compreender melhor e demonstrar como os músculos são estimulados, de forma genérica, para se obter o movimento das articulações em seres humanos. O modelo de ativação muscular estudado aqui é o modelo de Hill, e este será acoplado de um modelo cinemático da mão, ao aproximar os dedos para manipuladores robóticos de juntas rotacionais. O modelo de Hill é visto em outros estudos realizados sobre modelos musculares, como \cite{zajac1989muscle}, \cite{rosen1999performances}, \cite{burke2011motor} e \cite{winters1987biomechanical}, entre outros citados durante o trabalho. O modelo aqui implementado foi aproximado para uma versão normalizada, onde o objetivo era tornar este mais independente de parâmetros como o comprimento do músculo e força máxima, os quais variam por pessoa. Com um modelo mais simplificado se torna mais fácil a implementação em ferramentas como MATLAB e seu componente SIMULINK, sem perder a coerência dos resultados.

Serão introduzidas diferentes técnicas de miografia, invasivas e não-invasivas, para obtenção de dados de padrões musculares a fim de modelar o funcionamento músculo-esquelético do ser humano. Estas técnicas podem ser comparadas com o modelo desenvolvido em outros estudos para avaliar a confiabilidade do que foi desenvolvido. Ademais, será apresentado o modelo de Hill e suas particularidades, assim como um modelo cinemático para os dedos da mão humana utilizando os parâmetros de Denavit-Hartenberg.

A partir do modelo de Hill e do modelo cinemático, o trabalho apresenta uma análise do modelo de ativação muscular sobre cada articulação de um dedo anelar em um espaço tridimensional, utilizando ferramentas do MATLAB. E finalmente, são apresentados os resultados obtidos e como este estudo poderá expandir no futuro em outras pesquisas.