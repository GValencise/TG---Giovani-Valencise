\selectlanguage{Brazilian}

\chapter{RESULTADOS E DISCUSSÕES}\label{cap4}
A análise do resultado é feita a partir da validação da modelagem cinemática da mão e da ativação muscular. Em um primeiro momento estas são avaliadas separadamente, na seção \ref{resultado_cinematica} estão as avaliações feitas sobre os parâmetros de DH obtidos na seção \ref{cinematica_direta_mao} para condizerem com as restrições propostas na seção \ref{anatomia_mao}, e na seção \ref{resultado_ativacao} estão os gráficos de força, velocidade do músculo e ângulo da junta e como estes se relacionam ao proposto por \cite{zajac1989muscle}. Já na seção \ref{resultado_simulacao} está a integração entre os modelos apresentados e as discussões acerca dos resultados obtidos.
\section{Modelagem Cinemática da Mão}
\label{resultado_cinematica}
Para simplificação, nesta seção iremos abordar a modelagem dos dedos similares ao anelar, que possuem os mesmos parâmetros, diferente do dedo polegar. Levando em consideração as limitações da anatomia humana propostas por \cite{lin2000modeling} e os parâmetros de DH propostos na seção \ref{cinematica_direta_mao} para o dedo anelar é possível obter as relações entre os ângulos das articulações e a posição de cada junta enquanto o dedo se move. O modelo foi feito levando em consideração que este dedo possui uma junta diretiva e duas juntas flexivas, a representação dos ângulos $\theta$ estão nas figuras \ref{theta1} e \ref{thetas}. Nota-se como a representação das juntas é bem similar a apresentada por \cite{lee1995model} na figura \ref{Lee_Dedo}.

\begin{figure}[H]
\centering
\includegraphics[width = 0.6\textwidth]{img/theta1.jpg}
\caption[Representação do ângulo $\theta_1$]{Representação do ângulo $\theta_1$}
\label{theta1}
\end{figure}

\begin{figure}[H]
\centering
\includegraphics[width = 0.6\textwidth]{img/thetas.jpg}
\caption[Representação dos ângulos $\theta_2$, $\theta_3$ e $\theta_4$]{Representação dos ângulos $\theta_2$,$\theta_3$ e $\theta_4$}
\label{thetas}
\end{figure}

Correlacionando as juntas com as suas restrições, podemos destacar as restrições descritas para as juntas MP, PIP, DIP descritas na seção \ref{anatomia_mao}, e lembrando que o ângulo de flexão da junta DIP é relacionado ao ângulo da junta DIP como descrito em \ref{relacao_dip_pip}.
A partir dos parâmetros de DH é possível obter as matrizes de DH, onde pode-se obter as posições das juntas a partir dos ângulos em que as articulações estão dispostas. Mas é preciso determinar alguns parâmetros antes, como a distância entre as juntas, que, no dedo, seria representado pelo tamanho das falanges. Os parâmetros do tamanho das falanges foram obtidos através de medidas realizadas em algumas pessoas durante o projeto. Os parâmetros de tamanho de falange estão listados na tabela \ref{ParamResto} e as matrizes de DH para cada junta estão nas equações \ref{matriz_junta1}, \ref{matriz_junta2}, \ref{matriz_junta3} e \ref{matriz_junta4}.
As matrizes de cada junta em relação a junta anterior estão dispostas em \ref{junta1}, \ref{junta2}, \ref{junta3} e \ref{junta4} 
\begin{align}
A_1=
\begin{bmatrix}
cos_1 & 0 & -sen_1 & 0\\
sen_1 & 0 & cos_1 & 0\\
0 & -1 & 0 & 0\\
0 & 0 & 0 & 1\\
\end{bmatrix} \label{junta1}\\
A_2=
\begin{bmatrix}
cos_2 & -sen_2 & 0 & l1*cos_2\\
sen_2 & cos_2 & 0 & l1*sen_2\\
0 & 0 & 1 & 0\\
0 & 0 & 0 & 1\\
\end{bmatrix}  \label{junta2} \\
A_3=
\begin{bmatrix}
cos_3 & -sen_3 & 0 & l2*cos_3\\
sen_3 & cos_3 & 0 & l2*sen_3\\
0 & 0 & 1 & 0\\
0 & 0 & 0 & 1\\
\end{bmatrix}  \label{junta3} \\
A_4=
\begin{bmatrix}
cos_4 & -sen_4 & 0 & l3*cos_4\\
sen_4 & cos_4 & 0 & l3*sen_4\\
0 & 0 & 1 & 0\\
0 & 0 & 0 & 1\\
\end{bmatrix} \label{junta4}
\end{align}

\begin{align}
A_{MP_{AA}}= A_1 \label{matriz_junta1}\\
A_{MP_{F}}= A_1 * A_2 \label{matriz_junta2} \\
A_{PIP_{F}}= A_1 * A_2 * A_3 \label{matriz_junta3} \\
A_{DIP_{F}}= A_1 * A_2 * A_3 * A_4 \label{matriz_junta4}
\end{align}

Sendo $cos_i$ e $sen_i$ as representações de $cos(\theta_i)$ e $sen(\theta_i)$, e lembrando que $\theta_4 = \frac{2}{3}\theta_3$.

Com as tabelas é possível obter o posicionamento da junta em qualquer instante de tempo, dado os ângulos das juntas, estes provenientes do sistema de forças introduzido pelo modelo de Hill \cite{hill1938heat}. O posicionamento das juntas na tabela é descrito pela última coluna, onde as posições em cada eixo cartesiano é representado em uma linha, na linha 1 fica o valor do eixo x, na linha 2 fica o valor do eixo y e na linha 3 fica o valor do eixo z.

Para validar a restrição da angulação entre as juntas PIP e DIP foi simulado a resposta do dedo ao mover apenas a junta DIP, com uma entrada mostrada na figura \ref{ang_sim_pip}. A resposta para esta entrada está na figura \ref{sim_pip}, e é possível notar como a ângulação da junta DIP se move conforme a junta PIP porém com valores menores. Os ângulos para esta simulação foram retirados de medição empírica sobre dedos de algumas pessoas.

\begin{figure}[H]
\centering
\includegraphics[width = 1\textwidth]{img/angulacao_pip.jpg}
\caption[Ângulos de simulação de entrada para a junta PIP]{Ângulos de simulação de entrada para a junta PIP}
\label{ang_sim_pip}
\end{figure}

\begin{figure}[H]
\centering
\includegraphics[width = 1\textwidth]{img/simpip.png}
\caption[Simulação de entrada para a junta PIP]{Simulação de entrada para a junta PIP}
\label{sim_pip}
\end{figure}

O mesmo é feito para a junta MP, onde foram aplicados os ângulos de entrada para os músculos de flexão e abudção/adução descritos na figura \ref{ang_sim_mp}. A resposta para esta entrada está na figura \ref{sim_mp}, onde é possível notar a movimentação das duas juntas e como o dedo se comporta quando estes músculos estão sobre força. Lembra-se que o ângulo de adução/abdução da junta MP é limitado em um range de $-15^\circ$ à $15^\circ$, por isso sua variação angular é pequena perto do ângulo de flexão. Os ângulos para esta simulação foram retirados de medição empírica sobre dedos de algumas pessoas.

\begin{figure}[H]
\centering
\includegraphics[width = 1\textwidth]{img/angulacao_mp.jpg}
\caption[Ângulos de simulação de entrada para a junta MP]{Ângulos de simulação de entrada para a junta MP}
\label{ang_sim_mp}
\end{figure}

\begin{figure}[H]
\centering
\includegraphics[width = 1\textwidth]{img/simmp.png}
\caption[Simulação de entrada para a junta MP]{Simulação de entrada para a junta MP}
\label{sim_mp}
\end{figure}

E por fim, uma simulação de todas as juntas estimuladas, demonstrando que as matrizes de DH representam bem o movimento de um dedo para as entradas da figura \ref{ang_sim_DH}. A figura \ref{sim_DH} representa a movimentação do dedo quando a entrada é a da figura \ref{ang_sim_DH}.

\begin{figure}[H]
\centering
\includegraphics[width = 1\textwidth]{img/angulacao_tudo.jpg}
\caption[Ângulos de simulação de entrada para todas as juntas]{Ângulos de simulação de entrada para todas as juntas}
\label{ang_sim_DH}
\end{figure}

\begin{figure}[H]
\centering
\includegraphics[width = 1\textwidth]{img/sim_total.png}
\caption[Simulação de entrada para todas as juntas]{Simulação de entrada para todas as juntas}
\label{sim_DH}
\end{figure}

\section{Modelagem da Ativação Muscular}
\label{resultado_ativacao}
Como explicado em \cite{zajac1989muscle} existem relações entre o nível de ativação muscular e o ângulo das articulações. Estas relações podem ser modeladas como foi proposto por \cite{feng1999surface} e mostrado na seção \ref{forcas_modelo_simplificado} é possível criar um modelo no \textit{SIMULINK} capaz de aproximar estas resoluções. O objetivo desta simulação é colher os dados de saída do modelo feito no \textit{SIMULINK} e compará-los aos de outros estudos como de \cite{zajac1989muscle}, \cite{rosen1999performances} e \cite{feng1999surface}. Para esta análise foi necessário definir algumas constantes empíricas para o funcionamento do modelo, como o MJG (da seção \ref{forcas_modelo_simplificado}) e os ângulos de repouso das juntas. Naturalmente, como evidenciado em \cite{lin2000modeling} os dedos da mão não ficam totalmente alinhados com o resto da mão quando em repouso, isso justifica as saídas de angulação da junta que não partem do $0^\circ$. 

Para este estudo o valor de MJG para as diferentes falanges está disposto na tabela \ref{MJG_tabela}, os diferentes valores foram obtidos de forma empírica enquanto se fazia o estudo, já que não foram encontrados valores conclusivos na literatura. O valor da geometria das articulações varia conforme as falanges devido, justamente, a geometria de cada falange, seu comprimento e circunferência são fatores que demonstram as diferenças neste parâmetro.

\begin{table}[H]
\centering
\caption{Tabela de Valores para MJG}
\label{MJG_tabela}
\begin{tabular}{|c|c|c|}
	\hline
    Falange & Valor de MJG (m) \\ \hline
    Falange próxima & 20 \\ \hline
    Falange média & 10 \\ \hline
    Falange distal & 5 \\
	\hline
\end{tabular}
\end{table}

Este valor, como proposto por \cite{feng1999surface} é o ganho que converte os torques exercidos pelos músculos em ângulos, velocidades e acelerações nas articulações.

Nesta simulação são feitos os movimentos de flexão e extensão de um músculo. Como este é um modelo normalizado, as respostas esperadas para estes movimentos podem ser aplicadas para os outros músculos dos dedos sem perda de coerência dos dados. Portanto esta simulação é feita sobre a falange próxima de um dedo da mão. Primeiramente, é executada a ação de flexionar o músculo colocando como entrada um nível de ativação máximo igual a 1, já que o modelo é normalizado. Assim que a ativação do movimento de flexão cessa, se inicia a ativação do músculo de extensão de mesma intensidade. O uso destes dois músculos foi previsto na seção \ref{integracao_validacao}, e ambas as forças dos músculos são dadas como positivas e então o sinal é alterado conforme o músculo atua como avanço ou recuo.

Os sinais utilizados para realizar os movimentos de flexão e extensão estão ilustrados na figura \ref{ativacao_musculos}.

\begin{figure}[H]
\centering
\includegraphics[width = 1\textwidth]{img/niveis_ativacao.png}
\caption[Níveis de Ativação para Flexão e Extensão para os Músculos de uma Falange Próxima]{Níveis de Ativação para Flexão e Extensão para os Músculos de uma Falange Próxima}
\label{ativacao_musculos}
\end{figure}

Ao excitar o músculo de flexão com um nível de ativação $U=1$ espera-se uma resposta condizente com as propostas por \cite{rosen1999performances}, onde a força do elemento CE depende do nível de atuação e a força do elemento PE não depende do nível de atuação, e sim do comprimento do músculo \cite{zajac1989muscle}. De início é feito a simulação como se só houvesse o movimento de flexão do músculo, para analisar qual é a resposta dos elementos de força do músculo e como seria a saída do modelo, em velocidade e ângulo. A figura \ref{sim_flexao} mostra o resultado da simulação de flexão pura. É possível notar que o elemento CE do músculo de flexão é ativado conforme o nível de ativação, e uma vez que este cessa o elemento CE não apresenta mais força, porém o elemento PE que possui sua saída de força atrelada ao comprimento do movimento, continua apresentando uma saída de força, justamente por que não houve uma extensão do músculo após a flexão, ou seja, o músculo continua "tensionado". É possível notar o mesmo comportamento no comprimento do músculo e na angulação de saída da junta, eles permanecem estáticos mesmo após o nível de atuação do movimento de flexão ter acabado.

\begin{figure}[H]
\centering
\includegraphics[width = 1\textwidth]{img/flexao.png}
\caption[Resposta ao Movimento de Flexão para uma Falange Próxima]{Resposta ao Movimento de Flexão para uma Falange Próxima}
\label{sim_flexao}
\end{figure}

Ao mesmo passo que o nível de ativação foi utilizado para uma flexão pura, é feito uma extensão no músculo assim que a flexão acaba. Com isso os níveis de atuação utilizados passam a ser exatamente os dois da figura \ref{ativacao_musculos}, já que antes era somente o primeiro destes. A resposta para um movimento de flexão e então extensão está ilustrado na figura \ref{sim_flexao_extensao}. Nesta resposta é possível notar que o movimento de extensão faz com que o comprimento do músculo volte a 0 e assim o elemento PE passa a não apresentar mais uma resposta de força. Sendo assim os outros parâmetros de saída do modelo acompanham e apresentam uma curva plausível com o movimento físico dos músculos.

\begin{figure}[H]
\centering
\includegraphics[width = 1\textwidth]{img/flexao_extensao.png}
\caption[Resposta ao Movimento de Flexão e em seguida Extensão para uma Falange Próxima]{Resposta ao Movimento de Flexão e em seguida Extensão para uma Falange Próxima}
\label{sim_flexao_extensao}
\end{figure}

Logo nota-se que, primeiramente o elemento CE responde diretamente ao nível de atuação, e a medida que o comprimento do músculo varia o elemento PE responde com uma força proporcional. Os maiores valores de $\theta$ são acompanhados do aumento da velocidade e normalmente ocorrem logo após estes, o que demonstra uma robustez física do modelo, assim como o fato de os movimentos opostos resultarem em um comprimento final tendendo a 0 para o músculo. 

Com um movimento completo de flexão e extensão, é possível correlacionar as curvas de ângulo da junta e velocidade com as obtidas por \cite{rosen1999performances}, demonstradas na figura \ref{rosen_ativacao}. Lembra-se que o estudo performado por \cite{rosen1999performances} se tratava de uma captura de dados utilizando a tecnologia sEMG, a qual possui ruídos, os quais não existem nestas simulações, já que foram feitas utilizando curvas ideais produzidas pelo \textit{MATLAB}.

\begin{figure}[H]
\centering
\includegraphics[width = 1\textwidth]{img/Rosen1999_ativacao.JPG}
\caption[Resposta ao Movimento de Flexão seguido de Extensão em um Ombro]{Registros de uma sessão típica. Cinemática do ombro : ângulo da articulação do ombro, velocidade angular, e aceleração angular \cite{rosen1999performances}}
\label{rosen_ativacao}
\end{figure}

\section{Simulação do Movimento Completo}
\label{resultado_simulacao}

\subsection{Movimento Completo}